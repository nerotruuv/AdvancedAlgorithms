\documentclass{article}
\usepackage{graphicx} 
\usepackage[dutch]{babel}
\begin{document}
\sffamily
\begin{titlepage}
  \centering
    \vfill
    {\bfseries\Huge
      Verslag Tinlab Advanced Algorithms \\
        \vskip2cm
      }
      {\bfseries\Large
        S. T. Udent\\
      }
      {
        \bfseries\normalsize
        176-671\\
        \vskip1cm
        \today\\
    }    
    \vfill
    \includegraphics[width=4cm]{logohr.png} % also works with logo.pdf
    \vfill
    \vfill
\end{titlepage}
\newpage
\tableofcontents

\newpage
\section{Inleiding}
Zie hier een referentie naar Royce~\cite{royce1987managing} en nog een naar Clarke~\cite{modelchecking}\ldots 

\subsection{Sketch}
Veiligheid \\
Capaciteit \\
Beschikbaarheid \\
Efficientie \\
.. \\

\newpage
\subsection{Requirements}
\begin{description}
\item [Requirement 1] Afmetingen van de schutsluis moeten zodanig groot zijn dat de grootste schip dat nu of in de toekomst
gebruik maakt van de vaarweg de sluis kan paseren.
\item [Requirement 2] De capaciteit schutsluis zodat tijdsverlies wordt voorkomen doort maximaal aantal schepen dat
per tijdseenheid kan worden geschut.
\item [Requirement 3] Ook bij kleinere schepen tijdsverslies moet gewoon toi zijn minimum worden gebracht biij mogelijke vaarweg met grote
scheepvaartintensiteit.
\item [Requirement 4] Op elke moment moeten schippers/schepen door middel van een lichtsignaal de beschikbaarheid van de gebruik
van de schutsluit kunnen zien.
\item [Requirement 5] Ten alle tijden dient een schip die gebestigd is in een schutkolk uitsluitend met gesloten sluisdeuren van het
ene naar het ander water niveau te worden gebracht.
\item [Requirement 6] Indien sluisdeuren aan één kant in de open toestand zich bevinden zonder dat er een schip
in de schutkolk aanwezig is dient een lichtsignaal aan die kant te worden afgegeven dat aangeeft dat de schutsluis beschikbaar is voor gebruik.
\item [Requirement 7] Indien sluisdeuren in de gesloten toestand zich bevinden of in de open toestand waarbij een schip nog
aanwezig is in de schutkolk dient een lichtsignaal te worden afgegeven dat aangeeft dat de schutsluit niet beschikbaar is voor gebruik.
\end{description}

\newpage
\subsection{specificaties}

\subsection{Het vier variabelen model}
\subsubsection{Monitored variabelen}
\subsubsection{Controlled variabelen}
\subsubsection{Input variabelen}
\subsubsection{Output variabelen}

\subsection{Rampen}

\subsubsection{Ramp 1}
\begin{description}
\item[Beschrijving]
\item[Datum en plaats] 
\item[Oorzaak]
  %Beschrijf wat er mis ging in termen van het vier variabelen model/requirements/specificaties
\end{description}

\subsubsection{Ramp 2}
\begin{description}
\item[Beschrijving]
\item[Datum en plaats] 
\item[Oorzaak]
  %Beschrijf wat er mis ging in termen van het vier variabelen model/requirements/specificaties
\end{description}

\subsubsection{Ramp 3}
\begin{description}
\item[Beschrijving]
\item[Datum en plaats] 
\item[Oorzaak]
  %Beschrijf wat er mis ging in termen van het vier variabelen model/requirements/specificaties
\end{description}

\subsubsection{Ramp 4}
\subsubsection{Ramp 5}
\subsubsection{Ramp 6}

\section{Modellen}

\subsection{De Kripke structuur}

\subsection{Soorten modellen}

\subsection{Tijd}

\subsection{Guards en invarianten}

\subsection{Deadlock}

\subsection{Zeno gedrag}

\section{Logica}

\subsection{Propositielogica}

\subsection{Predicatenlogica}

\subsection{Kwantoren}

\subsection{Dualiteiten}

\section{Computation tree logic}

\subsection{De computation tree}

\subsection{Operator: AG}

\subsection{Operator: EG}

\subsection{Operator: AF}

\subsection{Operator: EF}

\subsection{Operator: AX}

\subsection{Operator: EX}

\subsection{Operator: p U q}

\subsection{Operator: p R q}

\subsection{Fairness}

\subsection{Liveness}

\newpage

\newpage
\bibliography{references}
\bibliographystyle{plain}
\end{document}


