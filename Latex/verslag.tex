\documentclass{article}
\usepackage{graphicx} 
\usepackage[dutch]{babel}
\usepackage{xcolor}
\begin{document}
\sffamily
\begin{titlepage}
  \centering
    \vfill
    {
        \bfseries\Huge Verslag Tinlab Advanced Algorithms \\
        \vskip2cm
        }
        {\bfseries\Large
          R. Karajev, M. Steijger\\
        }
        {
          \bfseries\normalsize
          0851997 0938713\\
          \vskip1cm
          \today\\
    }    
    \vfill
    \includegraphics[width=4cm]{logohr.png} % also works with logo.pdf
    \vfill
    \vfill
\end{titlepage}

\newpage
\tableofcontents


%==================================================================================
\newpage
\section{Inleiding}
    Voor Tinlab Advanced Algorithms wordt het methode Model Checking gegeven~\cite{modelchecking}. 
    Dit onderwerp wordt getoetst door middel van een eindopdracht. Het doel van deze 
    opdracht is een model aan te leveren van een compleet geautomatiseerd sluis. 
    Daarbij is het de bedoeling dat op basis van opgestelde requirements bepaalde 
    eigenschappen binnen dit model geverifieerd worden. Dit verslag word geschreven 
    om de keuzes die tijdens het process zijn gemaakt te onderbouwing en vast te leggen.
    Hierin is ook te lezen alles over de model en de werking ervan. Tot slot wordt
    aan het eind van dit verslag geverifieerde eigenschappen gepresenteerd.

%----------------------------------------------------------------------------------
\newpage
\section {Requirements}
    Voor deze opdracht is gekozen om een model te maken van een schutsluis.
    Als begin is het handboek voor het ontwerpen van schutsluizen bekeken.~\cite{ontwerpSchutsluizen}.
    Gezien de wens van de opdrachtgever voor het aanleveren van een model van een
    geautomatiseerde sluis is er voor gekozen om de volgende functionele eis dat in 
    het handboek gevonden is als uitgangspunt te nemen voor het opstellen van de 
    requirements voor deze opdracht. \\

    \textit{"De functionele eisen van een schutsluis zijn primair bedoeld voor de scheepvaart.}
    \textit{Als belangrijkste algemene functionele eis geldt dat de scheepvaart zodanig \textbf{snel} en 
    \textbf{veilig} de sluis moet kunnen passeren."} \\

    \noindent Voor deze opdracht wordt dus ook een schutsluis gemodelleerd die zo velig mogelijk functioneert zonder dat deze de scheepvaart belemert.
    Om dit te kunnen doen is hierbij als eerst gekeken naar een document waarbij algemeen 
    opbouw van Julianasluis 2 beschreven wordt om een opbouw van een schutsluis beter te begrijpen~\cite{bedienininstructie}. 
    Op basis van deze opbouw is er voor gekozen om een simplistische model ervan te bouwen. 
    Een simplistische model wordt gebruikt om ervoor te zorgen dat bij het modeleren de focus 
    vooral ligt op de kernfunctionaliteiten. Zo kunnen de veiligeisen die betrekking hebben 
    tot de functionaliteit dan ook geverifieerd worden. \\
    
    \noindent Voordat een model opgesteld wordt, wordt er als 
    eerst naar een functionele omschrijving sluitprocess gekeken te vinden in artikel~\cite{functioneel}. 
    Daaruit is informatie verkregen over functionering van een sluisprocess. In dit
    document zijn ook aanvullende functionele eisen gevonden wat betreft het 
    veiligheid. Deze eisen worden gebruikt als basis om te indentificeren welke 
    risico's er tijdens een sluitprocess bestaan. Tot slot wordt met behulp van
    deze informatie eigen veiligheideisen geformuleerd.
    
    \newpage
    \subsection{Veiligheidseisen}
        \textit{Sluisdeuren}
        \begin{enumerate} 
            \item Ten alle tijden mag er maar één sluisdeur open zijn. \\
            (\textit{De reden hiervoor is zodat het waterhoogteverschil altijd behouden wordt.})
            \item Een sluisdeur mag alleen open als de waterniveau aan beiden 
            kanten van de sluisdeur gelijk is. \\
            (\textit{Dit is een belangrijke veiligheideis om ongecontroleerde water verplaatsing te voorkomen, waarbij mogelijk letsel zou kunnen onstaan.})
            \item Ten alle tijden mag een deur alleen aangedreven worden als het
            invaarsein het sein 'verboden voor doorvaart' of 'sperren' aangeeft.
            \item Tijdens het nivelleren dienen ten alle tijden de sluisdeuren 
            gesloten te zijn en scheepvaartseinen het sein 'verboden voor doorvaart'
            aangeven. \\
            \textit{(Dit is nodig om ongeschaad een schip van een waterhoogteniveau naar een ander
            waterhoogteniveau te brengen.})
        \end{enumerate}

        \noindent\textit{Ricketschuiven}
        \begin{enumerate}
            \item Ten alle tijden mag een ricketschuif alleen open als de ricketschuif
            in de tegenovergestelde sluisdeur gesloten is. \\
            (\textit{Dit voorkomt de situatie waarbij er verbinding is tussen hoogland en laagland water.})
            \item Ten alle tijden mag een ricketschuif alleen open als de tegenovergestelde
            sluisdeur gesloten is. \\
            (\textit{Dit zorgt ervoor dat waterhoogteniveau veilig aangepast kan worden.})
        \end{enumerate}

        \noindent\textit{Scheepvaartseinen}
        \begin{enumerate}
            \item Ten alle tijden mag er maar aan één kant een scheepvaartsein het sein
            'vrij voor doorvaart' aangeven. \\
            (\textit{Dit voorkomt miscommunicatie en zorgt voor snel en veilige scheepvaart verkeer.})
            \item Ten alle tijden mag het scheepvaartsein 'vrij voor doorvaart' alleen 
            worden afgegeven als de betrokken sluisdeur volledig geopent is. \\
            (\textit{Dit zorgt ervoor dat schepen veilig het sluis binnen kunnen varen of verlaten zonder mogelijke ongelukken.})
        \end{enumerate}

        \newpage
        \noindent\textit{Noodstop}
        \begin{enumerate}
            \item Ten alle tijden kan een noodstop worden ingeschakkeld. \\
            (\textit{Noodstop zorgt ervoor dat alle componenten per direct gestopt/uitgezet worden.})
            \item Na het indrukken van de noodstop moet deze ook weer gereset kunnen worden.
            \item Sluisdeuren moet ten alle tijden tot een noodstop kunnen worden
            gebracht.
            \item Na het resetten van de noodstop dienen de sluisdeuren te hervatten
            met de onderbroken process. \\
            (\textit{Er is afweging gemaakt voor een simplistisch model,
            daarom wordt deze requirement dan ook niet geimplementeerd en dus ook niet geverifieerd.})
            \item Ricketschuiven moeten ten alle tijden tot een noodstop kunnen worden 
            gebracht waarbij de ricketschuiven per direct dienen te sluiten.
            \item Na het resetten van de noodstop dienen de ricketschuiven naar 
            hun originele positie te worden gebracht.
            \item Scheepvaartseinen kunnen ten alle tijden tot een noodstop worden
            gebracht waarbij het sein 'buiten gebruik' aangegeven wordt.
            \item Bij het resetten van de noodstop dienen de scheepvaartseinen
            het sein 'verboden voor doorvaart' aan te geven.
        \end{enumerate}

    \subsection{Efficientie}
        \begin{enumerate}
            \item Voordat een sluisdeur geopent wordt, dienen de ricketschuiven open 
            worden gemaakt. \\
            (\textit{Dit zorgt ervoor dat een deur minder weerstand ervaart. 
            Echter heeft deze requirement geen toegevoegde waarde om geimplementeerd te worden.})
            \item De bedieningstijd bij een schutprocess bedraagt niet langer dan 10 minuten.
        \end{enumerate}
%----------------------------------------------------------------------------------
\newpage
\section {Specificaties}
    Voor deze opdracht worden twee sluishoofden gemodelleerd zodat deze 
    samen een schutsluis vormen. Zo kan dan ook vooral de focus liggen bij 
    veilig functioneren. Per sluishoofd zijn de volgende keuzes gemaakt: \\
    \begin{itemize}
        \item \textit{Sluisdeur}
        \begin{itemize}
            \item Voor de sluisdeur is er gekozen om model te maken van puntdeuren.
            Dit zijn twee deuren die gebruikt kunnen worden voor scheiding tussen 
            de kolk en de buitenwater. Om dit model simpel te maken wordt hierbij 
            één deur gemodelleerd die het twee delige puntdeuren representeert.
            (\textit{Gezien identieke functionaliteit heeft het geen zin om twee 
            keer deze deur te gaan modeleren.})
            \item Een sluisdeur bestaat uit 4 toestanden en 1 noodtoestand.
            De toestanden zijn 'openen', 'open', 'sluiten', 'gesloten' en als
            laatste is er ook nog een 'noodtoestand'.
            \item Openen en sluiten van de sluisdeur wordt door middel van een timer
            bijgehouden. Het sluiten of openen van de sluisdeur duurt 1 minuut.
            \item Om aan \textit{Noodstop} requirement 4 te voldoen wordt het openen
            of sluiten van de sluisdeur bijgehouden door middel van een status variabele.
        \end{itemize}
        
        \item \textit{Ricketschuif}
        \begin{itemize}
            \item Voor de nivelleermiddel is er keuze gemaakt om een ricketschuif
            ervoor te modeleren. Een ricketschuif is een opening in een sluisdeur
            dat de waterpeil in de kolk kan controleren.
            \item Een ricketschuif bestaat uit 4 toestanden. De toestanden zijn 'openen', 
            'open', 'sluiten' en 'gesloten'. Gezien \textit{Noodstop} requirement 5 en 6
            worden 2 extra tussen toestanden gebruikt om de ricketschuif naar de betreffende
            toestand te kunnen brengen.
            \item Openen en sluiten van de ricketschuif wordt door middel van een timer
            bijgehouden. Het sluiten of openen van de ricketschuif duurt minder dan 1 minuut.
            \item Om aan \textit{Noodstop} requirement 5 en 6 te voldoen wordt het openen
            of sluiten van de ricketschuif bijgehouden door middel van een status variabele.
        \end{itemize}

        \item \textit{Scheepvaartsein}
        \begin{itemize}
            \item Voor de scheepvaartsein is gekozen om per sluisdeur twee scheepvaartseinen 
            te modeleren. Één voor buiten de sluis en één voor binnen de sluis. Hierbij wordt 
            één model voor twee verschillende scheepvaartseinen gebruikt met een variabele
            verschill in de 'noodtoestand' waar voor de binneste geld sein is 'uit' en de 
            buiteste het sein is 'buiten gebruik'. 
            \item De scheepvaartsein bestaat uit 3 toestanden en een 1 noodtoestand. Deze zijn
            als volgt gedefineerd:
                \begin{itemize}
                    \item 'buiten gebruik'/'noodtoestand' = Rood Rood / Uit
                    \item 'verboden voor doorvaart' = Rood
                    \item 'gereed maken voor doorvaart' = Rood Groen
                    \item 'vrij voor doorvaart' = Groen
                \end{itemize}
        \end{itemize}

        \item \textit{Noodstop}
        \begin{itemize}
            \item Voor de noodstop is gekozen om een model te maken van een drukknop die 
            na het drukken weer gereset moet worden naar zijn originele positie.
            \item Op het moment dat de noodstop is ingedrukt, moeten alle onderdelen per 
            direct in voor hun betreffende noodtoestand gaan.
        \end{itemize}

        \item \textit{Simulatie}
        \begin{itemize}
            \item Om een schutsluis te kunnen simuleren worden de besproken onderdelen gemaakt.
            Deze onderdelen worden gecombineerd met een simulatie voor scheepvaart om uiteindelijk
            tot een werkende model te komen die de werking van de schutsluis simplistisch uitbeeld.
            Dit wordt gedaan in de hoofdmodel waar alle logica van de schutsluis uitgewerkt wordt.
            \item Per kant worden 3 schepen geschut. Aantal schepen per kant wordt bijgehouden
            door middel van een getal. Elke keer dat een schip geschut is wordt deze getal verminderd met 1.
            \item Bij het begin toestand geldt voor alle onderdelen dat deze in toestand 'gesloten' bevinden.
            Voor scheepvaartseinen, deze dienen in de toestand 'verboden voor doorvaart' zich bevinden. 
            Bij het schutten van de schepen wordt de volgende process voor doorlopen:
                \begin{itemize}
                    \item Scheepvaartsein (laagland) in toestand 'gereed maken voor doorvaart'
                    \item Sluisdeur (laagland) openen
                    \item Scheepvaartsein (laagland) in toestand 'vrij voor doorvaart'
                    \item Sluisdeur (laagland) sluiten
                    \item Scheepvaartsein (laagland) in toestand 'verboden voor doorvaart'
                    \item Ricketschuif (hoogland) openen
                    \item Nivelleren
                    \item Ricketschuif (hoogland) sluiten
                    \item Scheepvaartsein (hoogland) in toestand 'gereed maken voor doorvaart'
                    \item 
                \end{itemize}
        \end{itemize}
    \end{itemize}
    
    
    is er voor gekozen om puntdeuren te modeleren. Puntdeuren zijn 
    twee deuren die gebruikt kunnen worden bij een ontworpen sluisdeur. Deze zorgen 
    voor scheiding tussen de kolk en de buiten water. Echter geeft die bij het 
    modeleren geen toegevoegde waarde gezien exact hetzelfde functionaliteit.
    Daarom wordt hier dan ook één sluisdeur per sluishoofd gemodelleerd.




    De schutsluis model wordt een simplistische model van Julianasluis 2. Deze model
    wordt opgebouwd uit volgende onderdelen:
    
    \begin{itemize}
        \item 2x sluisdeur onderdeel \\ 
        \item 2x scheepvaartsein onderdeel per sluisdeur onderdeel \\ (\textit{})
        \item 1x ricketschuif onderdeel per sluisdeur onderdeel (\textit{})
        \item 1x noodstop onderdeel (\textit{})
        \item 1x logische uitwerking (\textit{})
    \end{itemize}


%----------------------------------------------------------------------------------


\section {Modelcriteria op basis van Vaandragen}

%----------------------------------------------------------------------------------


\section {Gemodelleerde onderdelen}
    De schutsluis model wordt een simplistische model van Julianasluis 2. Deze model
    wordt opgebouwd uit volgende onderdelen:

    \begin{itemize}
        \item 2x sluisdeur onderdeel
        \item 2x scheepvaartsein onderdeel per sluisdeur onderdeel
        \item 1x ricketschuif onderdeel per sluisdeur onderdeel
        \item 1x noodstop onderdeel
        \item 1x logische uitwerking
    \end{itemize}


%----------------------------------------------------------------------------------


\section {Werking model}

%----------------------------------------------------------------------------------


\section {Geverifieeerde eigenschappen}
    \centering
    \begin{tabular}{ |p{4cm}||p{2cm}|p{5cm}|}
        \hline
            Requirements & Geverifieeerd & Formule \\
        \hline
        \hline
            Sluisdeuren 1 (veiligheid)          &           \leavevmode\color[HTML]{32CD32} PASS          &         Test        \\ \hline
            Sluisdeuren 2 (veiligheid)          &           \leavevmode\color[HTML]{32CD32} PASS          &         Test        \\ \hline
            Sluisdeuren 3 (veiligheid)          &           \leavevmode\color[HTML]{32CD32} PASS          &         Test        \\ \hline
            Sluisdeuren 4 (veiligheid)          &           \leavevmode\color[HTML]{32CD32} PASS          &         Test        \\ \hline
            Ricketschuiven 1 (veiligheid)       &           \leavevmode\color[HTML]{32CD32} PASS          &         Test        \\ \hline
            Ricketschuiven 2 (veiligheid)       &           \leavevmode\color[HTML]{32CD32} PASS          &         Test        \\ \hline
            Ricketschuiven 3 (veiligheid)       &           \leavevmode\color[HTML]{32CD32} PASS          &         Test        \\ \hline
            Scheepvaartseinen 1 (veiligheid)    &           \leavevmode\color[HTML]{32CD32} PASS          &         Test        \\ \hline
            Scheepvaartseinen 2 (veiligheid)    &           \leavevmode\color[HTML]{32CD32} PASS          &         Test        \\ \hline
            Noodstop 1 (veiligheid)             &           \leavevmode\color[HTML]{32CD32} PASS          &         Test        \\ \hline
            Noodstop 2 (veiligheid)             &           \leavevmode\color[HTML]{32CD32} PASS          &         Test        \\ \hline
            Noodstop 3 (veiligheid)             &           \leavevmode\color[HTML]{32CD32} PASS          &         Test        \\ \hline
            Noodstop 4 (veiligheid)             &           \leavevmode\color[HTML]{32CD32} PASS          &         Test        \\ \hline
            Noodstop 5 (veiligheid)             &           \leavevmode\color[HTML]{32CD32} PASS          &         Test        \\ \hline
            Noodstop 6 (veiligheid)             &           \leavevmode\color[HTML]{32CD32} PASS          &         Test        \\ \hline
            Noodstop 7 (veiligheid)             &           \leavevmode\color[HTML]{32CD32} PASS          &         Test        \\ \hline
            Noodstop 8 (veiligheid)             &           \leavevmode\color[HTML]{32CD32} PASS          &         Test        \\ \hline
            1 (efficientie)                     &           \leavevmode\color[HTML]{32CD32} PASS          &         Test        \\ \hline
            2 (efficientie)                     &           \leavevmode\color[HTML]{32CD32} PASS          &         Test        \\ \hline
        \hline
    \end{tabular}

%----------------------------------------------------------------------------------


%==================================================================================


\newpage
\bibliography{references}
\bibliographystyle{plain}
\end{document}


