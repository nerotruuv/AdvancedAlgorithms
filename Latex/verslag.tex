\documentclass{article}
\usepackage{graphicx} 
\usepackage[dutch]{babel}
\begin{document}
\sffamily
\begin{titlepage}
  \centering
    \vfill
    {
        \bfseries\Huge Verslag Tinlab Advanced Algorithms \\
        \vskip2cm
        }
        {\bfseries\Large
          R. Karajev, M. Steijger\\
        }
        {
          \bfseries\normalsize
          0851997 0938713\\
          \vskip1cm
          \today\\
    }    
    \vfill
    \includegraphics[width=4cm]{logohr.png} % also works with logo.pdf
    \vfill
    \vfill
\end{titlepage}

\newpage
\tableofcontents


%==================================================================================
\newpage
\section{Inleiding}
    Voor Tinlab Advanced Algorithms wordt het methode Model Checking gegeven. Dit
    onderwerp wordt getoetst door middel van een eindopdracht. Voor deze opdracht 
    moet een compleet geautomatiseerd model van een sluis worden gemaakt. Vervolgens 
    is het de bedoeling dat op basis van opgestelde requirements bepaalde eigenschappen 
    van het model geverifieerd worden. Het doel van dit verslag is het vastleggen 
    van keuzes die tijdens het process zijn gemaakt, onderbouwing geven op die keuzes,
    het presenteren van gemaakte model, uitleg geven over de werking van de model en
    tot slot presenteren van geverifieerde eigenschappen.

%----------------------------------------------------------------------------------
\newpage
\section {Requirements}
    Voor het ontwerpen van een sluis model is er als eerst gekeken naar een algemeen
    opbouw van Julianasluis 2~\cite{bedienininstructie}. Op basis van deze opbouw is
    er voor gekozen om een simplistische model ervan te bouwen met volgende
    onderdelen:
        \begin{itemize}
            \item 2x sluisdeur
            \item 2x scheepvaartsein per sluisdeur
            \item 1x ricketschuif per sluisdeur
            \item 1x noodstop
            \item 1x controle kamer
        \end{itemize}
    
    \noindent Voordat een model met opgenoemde onderdelen opgesteld wordt, wordt er als 
    eerst naar een functionele omschrijving sluitprocess gekeken~\cite{functioneel}. 
    Daaruit is een basis begrip verkregen van hoe zo'n sluis functioneert. In dit
    document zijn ook aanvullende functionele eisen gevonden wat betreft het 
    veiligheid. Deze eisen worden gebruikt als basis om te indentificeren welke 
    risico's er tijdens een sluitprocess bestaan. Tot slot wordt met behulp van
    deze informatie eigen veiligheideisen opgesteld.
    
    \subsection{Veiligheid}
        \textit{Sluisdeuren}
        \begin{itemize}
            \item Sluisdeuren moet ten alle tijden tot een noodstop kunnen worden
            gebracht.
            \item Na het resetten van de noodstop dienen de sluisdeuren te hervatten
            met de onderbroken process.
            \item Ten alle tijden mag er maar één sluisdeur open zijn.
            \item Een sluisdeur mag alleen open als de waterniveau aan beiden 
            kanten van de sluisdeur gelijk is.
            \item Ten alle tijden mag een deur alleen aangedreven worden als het
            invaarsein het sein "verboden voor doorvaart" of "sperren" aangeeft.
            \item Tijdens het nivelleren dienen ten alle tijden de sluisdeuren 
            gesloten te zijn en scheepvaartseinen het sein 'verboden voor doorvaart' 
            aangeven.
        \end{itemize}

        \noindent\textit{Ricketschuiven}
        \begin{itemize}
            \item Ricketschuiven moeten ten alle tijden tot een noodstop kunnen worden 
            gebracht waarbij de ricketschuiven per direct dienen te sluiten.
            \item Na het resetten van de noodstop dienen de ricketschuiven naar 
            hun originele positie te worden gebracht.
            \item Ten alle tijden mag er maar één ricketschuif open worden gemaakt.
            \item Ten alle tijden mag een ricketschuif alleen open als de tegenovergestelde
            sluisdeur gesloten is.
            \item Ten alle tijden mag een ricketschuif alleen open als de ricketschuif
            aan de tegenovergestelde sluisdeur gesloten is.
        \end{itemize}

        \noindent\textit{Scheepvaartseinen}
        \begin{itemize}
            \item Scheepvaartseinen kunnen ten alle tijden tot een noodstop worden
            gebracht waarbij het sein 'buiten gebruik' aangegeven wordt.
            \item Bij het resetten van de noodstop dienen de scheepvaartseinen
            het sein 'verboden voor doorvaart' aan te geven.
            \item Ten alle tijden mag er maar aan één kant een scheepvaartsein het sein
            'vrij voor doorvaart' aangeven.
            \item Ten alle tijden mag het scheepvaartsein "vrij voor doorvaart" alleen 
            worden afgegeven als de betrokken sluisdeur volledig geopent is.
        \end{itemize}

        \noindent\textit{Noodstop}
        \begin{itemize}
            \item Ten alle tijden kan een noodstop worden ingeschakkeld.
            \item Na het indrukken van de noodstop moet deze uiteindelijk gereset worden.
        \end{itemize}

    \newpage
    \subsection{Efficientie}
        \begin{itemize}
            \item De schutting process mag niet langer dan 30 minuten duren.
            \item Een boot kan alleen geschut worden als de waterniveau binnen de
            sluis gelijk is aan de waterniveau waar de boot vandaan komt.
        \end{itemize}
%----------------------------------------------------------------------------------
\newpage
\section {Specificaties}


%----------------------------------------------------------------------------------


\section {Modelcriteria op basis van Vaandragen}

%----------------------------------------------------------------------------------


\section {Gemodelleerde onderdelen}

%----------------------------------------------------------------------------------


\section {Werking model}

%----------------------------------------------------------------------------------


\section {Geverifieeerde eigenschappen}

%----------------------------------------------------------------------------------


%==================================================================================


\newpage
\bibliography{references}
\bibliographystyle{plain}
\end{document}


